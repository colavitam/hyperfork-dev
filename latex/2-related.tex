% !TEX root = top.tex
% above command is so that compilation is always from top.tex
\section{Related Work} \label{sec:related}
Our work in this paper draws on concepts from several lines of past research.

\Paragraph{Virtualization Technology} Virtualization technologies simulate the
physical resources of hardware in a software environment. This allows multiple
full operating system and application stacks to be hosted on one physical
machine. Virtualization has numerous benefits, including improving utilization
for hosting providers who can pack many virtual machines onto one physical
machine, providing a strong isolation boundary between mutually untrusting
entities, and adding a layer of fault tolerance since VMs are not tied to a
specific set of hardware and can be moved if a physical machine fails. Modern
instruction set extensions from Intel and AMD provide hardware-level support
for virtualization.~\cite{virt-techniques}

A hypervisor is an entity which runs with full privilege on the host system and
coordinates virtual machine guests. The Xen hypervisor~\cite{xen}, which is
widely used in the research community, takes the place of a host operating
system, and coordinates paravirtualized guests\footnote{The
\emph{paravirtualization} technique allows all guest instructions to run on the
bare metal, but requires modifications to the guest operating system. This is
contrasted with \emph{full virtualization}, where every instruction is emulated
by the hypervisor, or \emph{trap-and-emulate}, where only privileged
instructions are emulated.} through its special guest called Domain-0. The Linux
kernel supports virtualization through the module KVM~\cite{kvm}, which can be
used through user-mode virtual machine monitors such as QEMU~\cite{qemu},
kvmtool~\cite{kvmtool}, or Amazon's Firecracker~\cite{firecracker}.

\Paragraph{Cold-start Reduction Efforts} A number of different techniques have
been proposed to reduce cold-start latencies. One broad technique is to reduce
the size and complexity of the VM images which are used for serverless
functions. Because serverless functions in a cloud infrastructure run on
a fairly limited and standard set of physical machines, the number of
specialized drivers and kernel modules required to support this hardware is
vastly smaller than that in a standard distribution.  Additionally, because the
function will only need to run a single process, a number of other optimizations
can be made to reduce kernel size and boot time. Amazon utilizes such a stripped
down kernel in their Firecracker VMM~\cite{firecracker}. These kinds of
operating system optimizations have also been explored previously in a more
extreme form, with Unikernels~\cite{unikernels} and the Denali OS~\cite{denali},
where operating system features are treated as dependencies of userland programs
and included conditionally.

Other approaches to efficient resource isolation put the isolation boundary
above the operating system level. Containerization technologies such as
Docker~\cite{docker} utilize Linux cgroups~\cite{cgroups} and the
seccomp~\cite{seccomp} mechanism to isolate processes. While these offer very
fast startup times compared to virtual machines, they are a weaker form of
isolation, as processes in containers still share a kernel.

Another technique for reducing VM creation latencies involves the process of
\emph{flash-cloning}, in which new VMs are cloned from existing reference VMs.
This is an analogous process to forking processes within Linux. One work which
had success with this technique was the Potemkin Virtual
Honeyfarm~\cite{potemkin}. Lightweight VMs were created as honeypots for
detecting exploits in the wild. Potemkin implements flash-cloning within the
Xen hypervisor~\cite{xen} along with copy-on-write memory sharing, but focuses
on VM density rather than start-up latency. We implement flash-cloning within a
KVM based infrastructure to more closely match with industry standard
serverless infrastructure and with the goals of minimizing start-up latency and
utilizing native copy-on-write functionality within the Linux kernel.

\Paragraph{VM Live Migration} Flash-cloning relates very closely to the more
mature virtualization technology of VM live
migration~\cite{post-copy-migration}\cite{snowflock}. Cloud infrastructure has
long required the ability to efficiently migrate VMs across physical hosts. VM
live migration in general contains a superset of the functionality needed for
flash-cloning, but is focused on efficiently supporting the migration of VM
state across a slow network with minimal interruption to the VM. For serverless
computing infrastructure, we are focused on extremely low latency and therefore
do not want the complex techniques used for live-migration. However, we rely on
features pioneered by live migration research in order to efficiently duplicate
virtual machine state.
