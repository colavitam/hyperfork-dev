% !TEX root = top.tex
% above command is so that compilation is always from top.tex
\section{Related Work} \label{sec:related}
Our work in this paper draws on concepts from several lines of past research.

% TODO: I'm not convinced the history in this paragraph adds anything to the paper
\Paragraph{Virtualization Technology} Machine virtualization technology is
a complex and diverse space. Classical Hypervisors/Virtual Machine Monitors
(VMMs) relied on the fundamental primitive of \emph{Trap-and-Emulate}, where
sensitive instructions in the guest would be trapped by the hardware, and
emulated safely within the VMM using shadow structures for privileged
state~\cite{classic-virt}. In x86 however, not all sensitive instructions are
privaleged, meaning some cannot be trapped, and other techniques must also be used
to enable virtualization. \emph{Full virtualization} of unmodified guest
kernels was enabled through binary translation, where all sensitive
instructions could be translated into privileged instructions.
\emph{Paravirtualization} used modifications to the guest operating systems to
ensure sensitive operations were trapped. Today, modern hardware architectures
include special virtualization instructions which remove the need for binary
translation, and additionally remove the need for performance critical shadow
structures using two-dimensional hardware page tables~\cite{virt-techniques}.
Current virtualization technologies offer a mix of all these techniques,
including binary translation for full nested virtualization using Oracle's
HVX~\cite{hvx}, paravirtualization with Xen~\cite{xen}, and classic
trap-and-emulate utilizing modern hardware extensions and the QEMU x86
emulator~\cite{qemu} within the Linux kernel with KVM~\cite{kvm}. Xen is highly
used within the research community because of its relatively simple
software-only techniques, and KVM is valued for its tight integration with the
Linux kernel.

% TODO: I feel like we already said all of this in the intro. Is there anything new to add here?
\Paragraph{Serverless Computing} Serverless computing has become an
increasingly important and desirable platform in the cloud ecosystem. Stemming
from the grand vision of computation as a utility, serverless computing offers
users the ability to run application code directly on a black-box
infrastructure. Serverless has the potential to offer an easy to program,
auto-scaling, cost efficient way to utilize cloud infrastructure for users,
without the need to manage machine provisioning and configuration or service
orchestration~\cite{berkeley-serverless}. Although there exist many popular
industry serverless computing platforms
today~\cite{lambda}\cite{gcf}\cite{azure-cf}\cite{openwhisk}, serverless is
still an active area of research, with many improvements to be
made~\cite{peeking}\cite{trilemma}\cite{steps-back}.

% TODO: we need to make the distinction between unikernels and Amazon-style microkernels more clear; we seem to conflate them here
\Paragraph{Cold-start Reduction Efforts} A number of different techniques have
been proposed to reduce cold-start latencies. One broad technique is to reduce
the size and complexity of VMs which are used for serverless functions. Because
serverless functions running in a cloud infrastructure run on a fairly limited
and standard set of physical machines, the number of specialized drivers and
kernel modules required to support this hardware is vastly smaller than that in
a standard distribution.  Additionally, because the function will only need to
run a single process, a number of other optimizations can be made to reduce
kernel size and boot time. Amazon utilizes such a stripped down kernel in their
Firecracker VMM~\cite{firecracker}. These kinds of operating system
optimizations have also been explored previously in a more extreme form, with
Unikernels~\cite{unikernels} and the Denali OS~\cite{denali}, where there is no
real division between operating system and application code.

% TODO: citations
Other approaches to efficient resource isolation put the isolation boundary
above the operating system level. Containerization technologies such as
Docker~\cite{docker} utilize Linux cgroups~\cite{cgroups} and seccomp mechanism
to isolate processes. While these offer very fast startup times compared to
Virtual Machines, they are a weaker form of isolation, as processes in
containers still share a kernel.

Another technique for reducing VM creation latencies involves the process of
\emph{flash-cloning}, in which new VMs are cloned from existing reference VMs.
This is an analogous process to forking processes within linux. One work which
had success with this technique was the Potemkin Virtual
Honeyfarm~\cite{potemkin}. Lightweight VMs were created as Honeypots for
catching security vulnerabilities in cloud applications. This work successfully
implements flash-cloning within the Xen hypervisor~\cite{xen} along with
copy-on-write memory sharing, but focuses on VM density rather than start-up
latency. We implement flash-cloning within a KVM based infrastructure to more
closely match with industry standard serverless infrastructure, Amazon
Firecracker~\cite{firecracker}, and with the goals of minimizing start-up
latency and utilizing native copy-on-write functionality within the Linux
kernel.

\Paragraph{VM Live Migration} Flash-cloning relates very closely to the more
mature virtualization technology of VM live
migration~\cite{post-copy-migration}\cite{snowflock}. Cloud infrastructure has
long required the ability to efficiently migrate VMs across physical hosts. VM
live migration in general contains a superset of the functionality needed for
flash-cloning, but is focused on efficiently supporting the migration of VM
state across a slow network with minimal interruption to the VM. For serverless
computing infrastructure, we are focused on extremely low latency and therefore
do not want the complex techniques used for live-migration. We rely on many
features pioneered by live migration research in order to efficiently duplicate
virtual machine state.
