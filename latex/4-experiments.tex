% !TEX root = top.tex
% above command is so that compilation is always from top.tex
\section{Experimental Evaluation} \label{sec:experiments}
This is the experiments section
First summarize the findings in an implicit way. For example: 
``In this section, we demonstrate that System X achieves Z, Y, K'' 
where Y, Z, K are our main contributions in the paper as well. 

\Paragraph{Experimental Setup}
The experimental setup includes information about the following 5 things:
(i) \textit{Experimental Platform}, i.e., the machine we used, 
(ii) \textit{Implementation}, i.e., whether it is a prototype system, based on an existing one, and any other relevant detail,
(iii) \textit{Configuration}, i.e., the system-specific configuration and tuning that might have been needed, and discuss the possible values for any parameter.
(iv) \textit{Workloads}, i.e., dataset characteristics, and query characteristics, and (unless the whole paper is a single experiment this here just gives a summary of the workloads used and each experiment later on gives the exact setup)
(v) \textit{Metrics}, typically latency and/or throughput with some details and the 
methodology.

For the experimental platform we do 
not want to use exactly the same phrasing in every paper, however, the content should 
be the same.

\Paragraph{Explaining Each Figure}
We first put the most important figures, i.e., the 
ones that have the most important results in terms 
of the contributions. Each figure should be a 
single message. Each figure comes with two 
paragraphs. One paragraph for the setup and one 
paragraph for the discussion. The set-up paragraph 
should start by saying ``In this experiment we 
show that... We setup this experiment as follows...''.

The result paragraph explains in detail why we see 
what we see. Explanations should be based on facts 
and logic. Numbers that back up the explanations 
should be provided whenever possible. The 
paragraph should finish by repeating the main 
message again. 


