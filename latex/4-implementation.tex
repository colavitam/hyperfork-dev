% !TEX root = top.tex
% above command is so that compilation is always from top.tex

\section{Implementation} \label{sec:implementation}

We now describe the architecture and implementation of Hyperfork, along with
our design decisions regarding the flash-cloning operation.

For a KVM-based virtualization platform, flash-cloning requires duplicating
several pieces of VM state. Where possible, we aim to make heavy use of the
Linux fork operation to duplicate this state, as it is highly optimized and
provides copy-on-write functionality for duplicated pages. Specifically, guest
memory will be shared between the parent and child processes, and will only be
copied if written to. However, the fork system call alone leaves kernelspace
management state unduplicated, cannot duplicate all userspace VMM threads, and
leaves userspace bookkeeping file descriptors pointing to the parent process's
KVM management structures, which are inaccessible in the child.

Duplicating kernel state can either be performed within the kernel, by
coping in-kernel data structures, or in userspace by extracting and saving
guest-specific state in memory before forking. We predict a kernel
implementation may have slightly higher performance, but for simplicity we
implement our flash-cloning entirely in userspace within kvmtool.\footnote{Note
that a kernel implementation would also need to update file descriptors and
their memory mappings. This creates quite a mess in practice.} We also add an
ad-hoc guest-to-host communication channel with an extension to kvmtool to
enable our experimental evaluations.

\subsection{Flash-Cloning Support in kvmtool}

Our userspace HyperFork implementation for kvmtool proceeds in a series of
phases. At a high level, these phases are a triggering event, pre-fork
extraction of kernel state, forking, and post-fork reconstruction of VMM
management state in the child.

First, the fork is triggered, either by an administrator invoking the
\texttt{FORK} IPC via the command line interface, or by the guest sending a
signal to the host indicating that it is ready to fork. In either case, the IPC
thread receives this signal, pauses the virtual machine, and calls the pre-fork
routines.

Because KVM state becomes inaccessible in the child process after the VMM has
forked, all kernel state that must be restored in the child needs to be
recorded by the parent before forking. Alternatively, this could be implemented
by IPC between the parent and child, in which the state is sent after the fork
is complete. We have adopted the former approach. The pre-fork routine thus
saves all individual vCPU state for all vCPUs (registers, interrupt
configuration), global vCPU state (interrupt configuration, clock), and then
locks all mutexes that must survive in the guest. Note that it is not necessary
for the pre-fork routine to save the memory of the virtual machine. As the
memory is mapped in the VMM process, it is unaffected by the fork system call
and remains accessible. It will, however, need to be remapped in the guest
following the fork.

Once the pre-fork routine is complete, kvmtool calls the system call fork. In
the parent, all of the locks acquired by the pre-fork routines are released and
execution proceeds. In the child, the post-fork routine is invoked, performing
the following to reconstruct necessary VMM state:

\begin{enumerate}
\item Acquire new file descriptors for the KVM device and virtual machine
\item Create new file descriptors for the vCPUs
\item Restore individual and global vCPU state\footnote{In the process of
restoring this state, we encountered a bug in how KVM handles setting the
control registers when they change whether the guest is in long mode. We intend
to investigate this further and report it if necessary.}
\item Replace all eventfds used for signaling
\item Create new threads to handle devices and the execution of each vCPU
\item Release all mutexes locked in the pre-fork routine, and replace all
condition variables\footnote{Condition variables must be replaced, as in many
pthread implementations they contain an internal mutex that cannot be locked in
the pre-fork routine. If this mutex is locked by another thread when the IPC
thread performs the fork, the mutex will be permanently locked in the child
process.}
\item Attach the terminal device to a new pseudo-terminal, or detach it to
accept no further input\footnote{Due to a bug in kvmtool, operating with a
pseudoterminal with no slave is not supported.}
\end{enumerate}

Once the post-fork routine is complete, the vCPUs begin executing in the child
and the flash-clone operation is complete.

\subsection{Guest-to-Host Signaling}

For guests with more complex forking behavior, the guest may need to inform the
host when it is ready to fork. For example, a virtual machine running a python
program may chose to fork on boot, after python has started, after modules are
loaded, or after further program initialization has completed. As the guest's
userspace state is very difficult to detect from the VMM, we implement a
rudimentary system for guest-to-host signaling.

The guest-to-host signal consists of sending a message over one of the
processor's ports. This allows for a simple and very efficient way to send
short messages to the host, without requiring any modifications to the guest
kernel. This functionality is accessible from userspace through the C standard
library. We define one message to indicate that the guest is ready to fork, and
one message to indicate that the guest has completed its task. To support our
evaluations, we include in the guest images a \texttt{fork} and \texttt{done}
executable that signal the two events which can be easily executed by guest
benchmarks. Further messages could easily be defined for a more complex
deployment.
