% !TEX root = top.tex
% above command is so that compilation is always from top.tex
\section{Related Work or Background} \label{sec:related}
Our work in this paper draws concepts from several lines of past research.

\Paragraph{Virtualization Technology} Machine virtualization technology is
a complex and diverse space. Classical Hypervisors/Virtual Machine Monitors
(VMMs) relied on the fundamental primitive of \emph{Trap-and-Emulate}, where
sensitive instructions in the guest would be traped by the hardware, and
emulated safely within the VMM using shadow structures for privaleged
state~\cite{classic-virt}. In x86 however, not all sensitive instructions are
privaleged, meaning they cannot be trapped, and other techniques must also be used
to enable virtualization. \emph{Full virtualization} of unmodified guest
kernels was enabled through binary translation, where all sensitive
instructions could be translated into privaleged instructions.
\emph{Paravirtualization} used modifications to the guest operating systems to
ensure sensitive operations were trapped. Today, modern hardware architectures
include special virtualization instructions which remove the need for binary
translation, and additionally remove the need for performance critical shadow
structures using two-dimensional hardware page tables~\cite{virt-techniques}.
Current virtualization technologies offer a mix of all these techniques,
including binary translation for full nested virtualization using Oracle's
HVX~\cite{hvx}, paravirtualization with Xen~\cite{xen}, and classic
trap-and-emulate utilizing modern hardware extensions and the QEMU x86
emulator~\cite{qemu} within the Linux kernel with KVM~\cite{kvm}. Xen is highly
used within the research community because of its relatively simple
software-only techniques, and KVM is valued for its tight integration with the
Linux kernel.

\Paragraph{Serverless Computing} Serverless computing has become an
increasingly important and desireable platform in the cloud ecosystem. Stemming
from the grand vision of computation as a utility, serverless computing offers
users the ability to run application code directly on a black-box
infrastructure. Serverless has the potential to offer an easy to program,
auto-scaling, cost efficient way to utilize cloud infrastructure for users,
without the need to manage machine provisioning and configuration or service
orchestration~\cite{berkeley-serverless}. Although there exist many popular
industry serverless computing platforms
today~\cite{lambda}\cite{gcf}\cite{azure-cf}\cite{openwhisk}, serverless is
still an active area of research, with many improvements to be
made~\cite{peeking}\cite{trilemma}\cite{steps-back}.

\Paragraph{Flash Cloning / MicroVMs / Coldstart reduction efforts}
Include Potemkin and Snowflock, as well as that short one about caching python
modules. Also maybe include unikernels/Denali/Firecracker.
~\cite{potemkin}\cite{snowflock}\cite{unikernels}\cite{denali}\cite{firecracker}

\Paragraph{VM Live Migration}
Cite some state-of-the-art research / survey, maybe some tools/implementations.
Focus text on explaining difference from flash cloning.~\cite{post-copy-migration}
